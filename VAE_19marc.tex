\documentclass[t]{beamer}
%\documentclass{article}
%\usepackage{beamerarticle}
%\usepackage{amssymb}
\usepackage[T1]{fontenc}
\usepackage[utf8]{inputenc}
\usepackage[magyar]{babel}
%\usepackage{graphics}
\usepackage{tikz}
\selectlanguage{magyar}
%
\input{hpbk_macros.tex}
\mode<presentation>
{
  \usetheme{default}
 \setbeamertemplate{headline}
  {\leavevmode
  \begin{beamercolorbox}[wd=.4\paperwidth,ht=2.25ex,dp=1ex,left]{section in head/foot}%
    \usebeamerfont{section in head/foot}\hspace*{2ex}\insertshorttitle
  \end{beamercolorbox}%
  \begin{beamercolorbox}[wd=.2\paperwidth,ht=2.25ex,dp=1ex,left]{date in head/foot}%
    \usebeamerfont{date in head/foot}\hspace*{2em}\insertauthor
  \end{beamercolorbox}%
  \begin{beamercolorbox}[wd=.4\paperwidth,ht=2.25ex,dp=1ex,right]{date in head/foot}%
    \usebeamerfont{date in head/foot}\insertsubtitle{}\hspace*{2em}
    \insertframenumber{} / \inserttotalframenumber\hspace*{2ex}
  \end{beamercolorbox}%
  }
  \setbeamertemplate{footline}
  {\leavevmode
  \begin{beamercolorbox}[wd=.5\paperwidth,ht=2.25ex,dp=1ex,left]{section in head/foot}%
    \usebeamerfont{section in head/foot}\hspace*{2ex}\insertinstitute
  \end{beamercolorbox}%
  \begin{beamercolorbox}[wd=.5\paperwidth,ht=2.25ex,dp=1ex,right]{date in head/foot}%
    \usebeamerfont{date in head/foot}\hspace*{2em}
  \end{beamercolorbox}%
  }
  \usecolortheme{crane}
  \usecolortheme{orchid}
  \useinnertheme[shadow]{rounded}
}
%

\title{\textbf{(Variational) Autoencoder}}

\subtitle{}

\author{Kiss Melinda Flóra}
\date{}

\begin{document}


\begin{frame}
  \titlepage
\end{frame}


\begin{frame}
  \frametitle{Autoencoder}

  Goal: reconstruct the input

  \begin{figure}
    \includegraphics[scale=0.12]{AE.png}
  \end{figure}

  Encoder:
  
  \begin{itemize}
  \item Maps the input to the latent space
  \item $h = f(x)$ 
  \end{itemize}

  Decoder:

  \begin{itemize}
  \item Reconstruct the input from the latent space
    representation
  \item $r = g(h)$
  \end{itemize}

  We want $\hat{x} = g(f(x))$ to be close to $x$.
\end{frame}

\begin{frame}
  \frametitle{Autoencoder}
  
  Autoencoding is a data compression algorithm where the
  compression and decompression functions are

  \begin{itemize}
  \item data-specific
  \item lossy 
  \item learned automatically from examples
  \end{itemize}

  \vspace{15pt}

  Not good for data compression:

  \begin{itemize}
  \item data-specific
  \item Very hard to train one, which does a better job than
    JPEG
  \end{itemize}

\end{frame}


\begin{frame}
  Principal component analysis vs autoencoder
  
  \begin{figure}
    \includegraphics[scale=0.3]{PCA_AE.png}
  \end{figure}
\end{frame}


\begin{frame}
  \frametitle{Constraints}

  We hope we can discover dependencies in the data. 
  
  \begin{figure}
    \includegraphics[scale=0.2]{autoencoder2.png}
  \end{figure}

  We need to constrain the model to prevent memorizing.
\end{frame}

\begin{frame}
  \frametitle{Autoencoder}

  An ideal autoencoder is

  \vspace{5pt}

  \begin{itemize}
  \item Sensitive enough to the inputs to acccurately build
    a reconstruction
  \item Insensitive enough so the model doesn't simply
    memorize the data. 
  \end{itemize}

  \vspace{15pt}

  In many cases the loss function consists of two parts:

  \vspace{5pt}

  \begin{itemize}
  \item Reconstruction loss $\mathcal{L}(x, g(f(x)))$
  \item Regularizer, which discourages memorizing. 
  \end{itemize}
\end{frame}


\begin{frame}
  \frametitle{Undercomplete autoencoder}

  Easiest way is to constrain the number of nodes in the
  hidden layers. 

  \begin{figure}
    \includegraphics[scale=0.2]{autoencoder.png}
  \end{figure}

  Limits the amount of information that can flow through the
  network.

  \vspace{2mm}
  
  No regularizer, only reconstruction loss. 
  
\end{frame}

\begin{frame}
  \frametitle{Sparse autoencoder}

  Regularized autoencoder: we have constraints, but not on
  the dimensions. 

  \begin{figure}
    \includegraphics[scale=0.2]{SparseAE.png}
  \end{figure}

  Only a few nodes are activated. 
\end{frame}


\begin{frame}
  \frametitle{Sparse autoencoder}

  Loss function: Reconstruction loss + sparsity constraint

  \vspace{15pt}

  Two types:

  \begin{itemize}
  \item $L_1$ regularization
  \end{itemize}

  \vspace{10pt}

  sparsity constraint = $\lambda \sum_{i}|a_i^{(h)}|$

  \vspace{10pt}

  $\lambda$ is a hyperprameter, $a_i^{(h)}$ is the activation of
  the $i$th node in the hidden layer.

  \vspace{10pt}

  \begin{itemize}
  \item Kullback-Leibler divergence
  \end{itemize}

\end{frame}

\begin{frame}
  \frametitle{Kullback-Leibler divergence}

  Measures distance between two distributions

  \[
  D_{\text{KL}}(p \| q) = \int p(x)\cdot \log
  \frac{p(x)}{q(x)}\ dx = \int p(x)\cdot \big(\log p(x) - \log
  q(x)\big)\ dx
\]

\[
  D_{\text{KL}}(p \| q) = E_{x\sim p} \log \frac{p(x)}{q(x)}
\]

\begin{itemize}
  \item Not symmetric
  \item Triangle inequality doesn't hold
  \item nonegative
  \end{itemize}

\end{frame}

\begin{frame}
  \frametitle{Sparse autoencoder}

  $\rho$: sparsity parameter: small number

  \vspace{5pt}

  For all nodes in the hidden layer

  \[
    \hat{\rho}_j= \frac{1}{m}\sum_{i=1}^m (a_j^{(h)}(x^{(i)}))
  \]

  Sparsity constraint:
  $\sum_{j=1}^{s_h} D_{\text{KL}}(\rho \| \hat{\rho}_j)$,
  Kullback-Leibler divergence berween two Bernoulli
  distributions
  \[
  D_{\text{KL}}(\rho \| \hat{\rho}_j) = \rho \log \frac{\rho
  }{\hat{\rho}_j} + (1-\rho) \log \frac{1-\rho}{1-
    \hat{\rho}_j}
\]

Loss function:

\[
  \mathcal{L}(x, \hat{x})+ \beta \sum_{j=1}^{s_h}
  D_{\text{KL}}(\rho \| \hat{\rho}_j),
\]
\end{frame}

\begin{frame}
  \frametitle{Denoising autoencoder}

  \begin{figure}
    \includegraphics[scale=0.25]{Denoising.png}
  \end{figure}

\end{frame}

\begin{frame}

\begin{columns}

    \begin{column}{0.5\textwidth}
      \begin{figure}
        \includegraphics[scale=0.2]{mnistAE.png}
      \end{figure}
    \end{column}

    \begin{column}{0.5\textwidth}
      
      \vspace{15pt}
      
      Training an autoencoder on the MNIST dataset, and
      visualizing the encodings from a 2D latent space
      reveals the formation of distinct clusters. This makes
      sense, as distinct encodings for each image type makes
      it far easier for the decoder to decode them. 
    \end{column}
    
  \end{columns}
\end{frame}


\begin{frame}
  \frametitle{Variational autoencoder}

  \textbf{Generative model:}

  \vspace{10pt}

  Goal: approximate the underlying $p^*(x)$ distribution of
  the training set.

  \vspace{10pt}

  Want to generate new samples, similar to the training
  set.

  \vspace{20pt}

  \textbf{Latent variable model:}

  \vspace{10pt}

  Learn a low-dimensional latent representation

  \vspace{10pt}

  We assume it generated the actual training data. 
\end{frame}

\begin{frame}
  \frametitle{Problem scenario}

  We have a dataset
  ${\mathcal{D}} = \{{\bf{x}}^{(i)}\}_{i=1}^n$ consisting of
  $n$ i.i.d samples from some continuous or discrete
  variable $x$.

  \vspace{2mm}
  
  We assume the data was generated by a random process,
  involving an unobserved continuous variable ${\bf{z}}$.

  \begin{enumerate}
  \item A value ${\bf{z}}^{(i)}$ is generated from some
    prior distribution $p({\bf{z}})$
    
    \vspace{2mm}
    
  \item A value ${\bf{x}}^{(i)}$ is generated from some
    conditional distribution
    $p({\bf{x}} \mid{\bf{z}})$.
  \end{enumerate}

  \vspace{2mm}

  \[
    p(x) = \int p(x|z) p(z)\ dz
  \]

  
  \vspace{2mm}

  We aim to maximize the probability of each $x^{(i)}$ in the
  training set.  

 \end{frame}


\begin{frame}
  We are interested in a general algorithm, which works in
  the case of

  \vspace{2mm}

  \begin{enumerate}
  \item Intractability: the marginal likelihood:
    \[
      p(x) = \int p(x\mid z) p(z) dz
    \]
    and the posterior density:
    \[
      p(z\mid x)= \frac{p(x, z)}{p(x)}
    \]
    are both intractable,

    \vspace{2mm}

  \item A large dataset: we have so much data that
    sampling-based solutions would be too slow.

    \[
      p(x) \approx \frac{1}{N}\sum_{i=1}^N p(x|z_i)
    \]
  \end{enumerate}
\end{frame}


\begin{frame}
  \frametitle{The prior and the likelihood}
  
   We assume that samples of $z$ can be drawn from a simple
  distribution, i.e $\mathcal{N}(0,I)$.

  \vspace{2mm}

  Also, we assume that the likelihood is from a parametric
  family with parameters $\theta$.
  
  \[
    p_\theta(x|z) = \mathcal{N}(x| f(z,\theta), \sigma^2
    \cdot I),
  \]
  so it is Gaussian with mean $f(z,\theta)$ and covariance
  $\sigma^2\cdot I$.

  \vspace{2mm}

  We want to optimize the $\theta$ parameters, so the
  logarithm of the marginal likelihood will be maximal:
  \[
    \log p_{\theta}(x^{(1)}, \dots, x^{(n)})= \sum_{i=1}^n
    \log p_{\theta}(x^{(i)}).
  \]


\end{frame}

\begin{frame}
  We could compute $p(x^{(i)})$ approximately, by sampling a
  lot of $z$ values, $z_1, \dots, z_\ell$, then compute $1/\ell
  \sum_j p(x^{(i)}| z_j) $, but usually the $\ell$ has to be
  very large.

  \vspace{2.5mm}

  
  In practice, for most $z$, the probability
  $p_{\theta}(x^{(i)}\mid z)$ will be almost zero, so they
  contribute nothing to $p_\theta(x^{(i)})$.

  \vspace{2.5mm}

  
  The key idea behind the variational autoencoder is that we
  would like to sample values of $z$ that are likely to have
  produced $x^{(i)}$.

  \vspace{2.5mm}

  So we want to infer the characteristics of $z$, but we can
  only see the $x^{(i)}$.

  \vspace{2.5mm}
  
  We cannot use the posterior $p_{\theta}(z\mid x^{(i)})$
  because is intractable, so we will approximate it with a
  new distribution, $q_{\phi}(z\mid x^{(i)})$.
 \end{frame}

\begin{frame}
  
  \begin{figure}
    \includegraphics[scale=0.35]{VAE.png}
  \end{figure}

  \vspace{2mm}

  If the space of $z$ values that are likely under
  $q_{\phi}$ is much smaller than those likely under $p(z)$,
  then we can compute
  $E_{q_{\phi}(z\mid x^{i})}(\log p_{\theta}(x^{(i)}\mid
  z))$ relatively easily.

  \vspace{2mm}

  We will relate
  $E_{q_{\phi}(z\mid x^{i})}(\log p_{\theta}(x^{(i)}\mid
  z))$ and $p_{\theta}(x^{(i)})$.
  
\end{frame}

\begin{frame}
  We compute the Kullback-Leibler divergence between
  $p_{\theta}(z\mid x^{(i)})$ and $q_\phi(z|x^{(i)})$:

  \vspace{1mm}
  
  \[
    D_{KL}(q_\phi(z| x^{(i)}) \| p_\theta(z| x^{(i)})) =
    E_{q_{\phi}(z|x^{(i)})}(\log q_\phi(z|x^{(i)})-\log
    p_{\theta}(z| x^{(i)}))=
  \]
  
  \vspace{1mm}
  
  Applying Bayes rule to $p_\theta(z\mid x^{(i)})$:

  \vspace{1mm}
  
  \[
    = E_{q_{\phi}(z|x^{(i)})}(\log q_\phi(z| x^{(i)})-\log
    p_{\theta}(x^{(i)}| z)-\log p(z))+ \log
    p_{\theta}(x^{(i)})
  \]

  \vspace{2mm}
  
  After rearranging:
  \[
    \log p_{\theta}(x^{(i)})- D_{KL}(q_\phi(z| x^{(i)}) \|
    p_\theta(z| x^{(i)})) =
  \]
  \[
    = E_{q_{\phi}(z| x^{(i)})}\big(\log p_{\theta}(x^{(i)}|
    z)+ \log p(z) - \log q_\phi(z| x^{(i)})\big)=
  \]
\end{frame}

 
\begin{frame}
  \frametitle{Variational lower bound}

   \[
    = E_{q_{\phi}(z| x^{(i)})}\log p_{\theta}(x^{(i)}| z) -
    D_{KL}(q_{\phi}(z| x^{(i)})\| p(z)).
  \]

  We call the right hand side the variational lower bound:

  \[
    \mathcal{L}(\theta, \phi, x^{(i)})= E_{q_{\phi}(z|
      x^{(i)})}\log p_{\theta}(x^{(i)}| z) -
    D_{KL}(q_{\phi}(z| x^{(i)})\| p(z)).
  \]

  This is indeed a lower bound for $\log p_{\theta}(x^{(i)})$,
  because the KL-divergence is always nonnegative.

  \vspace{2mm}
  
  We want to maximize $\log p_{\theta}(x^{(i)})$, so we need to
  optimize the lower bound with respect to both parameters,
  $\theta$ and $\phi$.

  \vspace{2mm}

  We want to perform stochastic gradient descent on the
  right hand side. 
\end{frame}

\begin{frame}
  \frametitle{Latent space}

  \begin{figure}
   \includegraphics[scale=0.29]{Latent.png}
  \end{figure}
\end{frame}

\begin{frame}
  The usual choice for $q_{\phi}(z| x^{(i)})$ is
  $\mathcal{N}(z\mid \mu_\phi(x^{(i)}),
  \Sigma_\phi(x^{(i)})$, where $\mu_\phi$ and $\Sigma_\phi$
  are arbitrary, deterministic functions that can be learned
  from data.
 
  \vspace{2mm}

  Now the KL-divergence on the right hand side is between
  two multivariate Gaussian distributions, which can be
  computed in closed form:

  \vspace{1mm}
  
  \[
    D_{KL}(\mathcal{N}(\mu_0, \Sigma_0)\| \mathcal{N}(\mu_1,
    \Sigma_1)) =
  \]
  \[
    = \frac{1}{2}\Big(\tr (\Sigma_1^{-1}\Sigma_0)+
    (\mu_1-\mu_0)^T \Sigma_1^{-1}(\mu_1-\mu_0)-k +\log
    \Big(\frac{\det \Sigma_1}{\det \Sigma_0}\Big)\Big),
  \]
  where $k$ is the dimensionality of the distribution.
\end{frame}
  
\begin{frame}
  In our case:
  \[
    D_{KL}(\mathcal{N}(\mu(x^{(i)}), \Sigma(x^{(i)}))\|
    \mathcal{N}(0,I)) =
  \]
  \[
    = \frac{1}{2}\Big(\tr (\Sigma(x^{(i)}))+ (\mu(x^{(i)})^T
    (\mu(x^{(i)})-k +\log
    \det(\Sigma(x^{(i)}))
  \]

  We could use sampling to estimate the other term on the
  right hand side, but instead as is standard in stochastic
  gradient descent we take one sample of $z$ and treat
  $p_\theta(x^{(i)}\mid z)$ for that $z$ as an approximation
  of
  $E_{q_{\phi}(z\mid x^{(i)})}\log p_{\theta}(x^{(i)}\mid
  z)$.

  \vspace{1mm}

  So we compute the gradient of
  \[
    \log p_{\theta}(x^{(i)}\mid z)- D_{KL}(q_{\phi}(z\mid
    x^{(i)})\| p(z)),
  \]
  and average it over arbitrarily many samples of $z$. 
\end{frame}

\begin{frame}
  \frametitle{Reparametrization trick}

  The problem with he last equation is that
  $E_{q_{\phi}(z\mid x^{(i)})}\log p_{\theta}(x^{(i)}\mid
  z)$ depends on the parameters $\phi$ also, but here the
  dependency has disappeared.

  \vspace{2mm}

  We need to backpropagate the error through a layer that
  samples $z$ from $q_{\phi}(z\mid x)$, which is a
  non-continuous operation and has no gradient.

  \vspace{2mm}

  Solution: we move the sampling to an input layer.

  \vspace{2mm}

  Given $\mu(x^{(i)})$ and $\Sigma(x^{(i)})$, tha mean and
  covariance of $q_{\phi}(z\mid x^{(i)})$, we can sample
  from $\mathcal{N}(\mu(x^{(i)}), \Sigma(x^{(i)}))$ by first
  sampling $\vep \sim \mathcal{N}(0,I)$, and then computing
  $z = \mu(x^{(i)})+ \Sigma^{1/2}(x^{(i)})\cdot \vep$.
\end{frame}

\begin{frame}
  So the equation becomes:

  \[
    E_{\vep \sim \mathcal{N}(0,I)}\Big(\log
    p_{\theta}(x^{(i)}\mid z = \mu(x^{(i)})+
    \Sigma^{1/2}(x^{(i)})\cdot \vep)-
  \]
  \[
    - D_{KL}(q_{\phi}(z\mid
    x^{(i)})\| p_{\theta}(z))\Big)
  \]

  
  \begin{figure}
   \includegraphics[scale=0.4]{ReparamTrick.png}
  \end{figure}
\end{frame}

\begin{frame}
  \frametitle{Testing}

  \begin{figure}
   \includegraphics[scale=0.25]{Output_mnist.png}
  \end{figure}
\end{frame}

\end{document}


 \vspace{1mm}

  In practice $\mu$ and $\Sigma$ are implemented via neural
  networks, and $\Sigma$ is constrained to be a diagonal
  matrix.


  
\begin{frame}
  \begin{itemize}
    \item Sensitive to the inputs enough to accurately build a
    reconstruction
    \item Insensitive enough to the inputs that the model
      doesn't simply memorize the data
  \end{itemize}

  \begin{figure}
   \includegraphics[scale=0.2]{autoencoder2.png}
  \end{figure}
\end{frame}

\begin{frame} 
  \frametitle{Autoencoder}

  The bottleneck is the key attribute, without it the
  network could easily memorize the input values.
  
  \begin{figure}
    \includegraphics[scale=0.25]{autoencoder.png}
  \end{figure}
\end{frame}

\begin{frame}
  \frametitle{Undercomplete autoencoder}

  We only constrain the number of nodes in the hidden
  layers. 

  \begin{figure}
   \includegraphics[scale=0.3]{UndercompleteAE.png}
  \end{figure}
\end{frame}

\begin{frame}
  \frametitle{Autoencoders}

  They are not good for data compression, it is very
  difficult to train an autoencoder that does a better job
  than a basic algorithm like JPEG. The fact that
  autoencoders are data-specific makes them generally
  impractical for real-world data compression problems.

  \vspace{3mm}

  Today two interesting practical applications of
  autoencoders are data denoising and dimensionality
  reduction for data visualization.

  \vspace{3mm}

  The fundamental problem with autoencoders, for generation,
  is that the latent space they convert their inputs to and
  where their encoded vectors lie, may not be continuous, or
  allow easy interpolation.
\end{frame}


